\documentclass[12pt]{article} % Set base font size to 12pt
\usepackage{graphicx}
\usepackage{geometry} % Adjust page margins
\usepackage{setspace} % Line spacing
\usepackage{titlesec} % Customize section titles
\usepackage{tocloft} % Customize table of contents
\usepackage{fancyhdr} % Headers and footers
\usepackage{hyperref} % Add hyperlinks
\usepackage{xcolor} % Color package
\usepackage[T1]{fontenc} % Proper font encoding
\usepackage[utf8]{inputenc} % UTF-8 encoding
\usepackage[ngerman]{babel} % German language support

% Set page margins
\geometry{a4paper, margin=2.5cm}

% Set line spacing
\setstretch{1.5} % Adjust line spacing for better readability

% Customize section titles
\titleformat{\section}[block]{\LARGE\bfseries\color{black}}{}{0em}{\filcenter}
\titlespacing*{\section}{0pt}{3.5ex plus 1ex minus .2ex}{2.3ex plus .2ex}

% Customize table of contents
\renewcommand{\cftsecleader}{\cftdotfill{\cftdotsep}}
\renewcommand{\contentsname}{Inhaltsverzeichnis}
\renewcommand{\cftaftertoctitle}{\par\nobreak\bigskip\bigskip\bigskip} % Add space after TOC title
\setlength{\cftbeforesecskip}{0.5em} % Adjust spacing between section entries
\setlength{\cftaftertoctitleskip}{2cm} % Adjust spacing between TOC title and entries
\hypersetup{
    colorlinks=true,
    linkcolor=blue,
    filecolor=magenta,
    urlcolor=cyan,
}

% Define headers and footers
\pagestyle{fancy}
\fancyhf{} % Clear default headers and footers
\fancyhead[R]{\thepage} % Page number on right side of header
\fancyhead[L]{\nouppercase{\leftmark}} % Chapter title on left side of header
\renewcommand{\headrulewidth}{0pt} % Remove header line
\fancyfoot[C]{\thepage} % Page number in the center of footer
\renewcommand{\footrulewidth}{0pt} % Remove footer line

% Define light gray color
\definecolor{lightgray}{RGB}{240,240,240}

\begin{document}

% Title Page
\begin{titlepage}
    \centering
    \vspace*{3cm}
    {\Huge\bfseries\textcolor{blue}{\MakeUppercase{ Studentleben }}\par} % Increased font size and colored title
    \vspace{0.5cm} % Adjust space between title and author
    {\Large\textit{ Maja Schmidt }\par} % Italic author name
    \vfill
    \includegraphics[width=0.9\textwidth]{ cover.jpg } % Larger cover image
    \vfill
    \today
\end{titlepage}

% Autorenvita
\section*{Autorenvita}
\vspace{4cm} % Adjust space between "Autorenvita" and "Inhaltsverzeichnis"
Maja Schmidt ist eine renommierte Autorin, die sich auf romantische Romane mit dem Schwerpunkt auf das Studentenleben spezialisiert hat. Mit ihrer einfühlsamen Schreibweise und ihrer Fähigkeit, emotionale Verbindungen zwischen den Lesern und den Charakteren herzustellen, hat sie bereits eine treue Leserschaft gewonnen. Maja Schmidt ist bekannt für ihre Fähigkeit, authentische und inspirierende Geschichten zu erzählen, die die Leser zum Nachdenken anregen und sie in fesselnde Welten eintauchen lassen.

% Place table of contents on a separate page
\clearpage
\tableofcontents
\clearpage

% Chapters

\section{ Ankunft in der neuen Stadt }
 Emma stand vor dem majestätischen Eingangstor der Sonnenstadt Universität und spürte eine Mischung aus Aufregung und Nervosität. Die prunkvollen alten Gebäude und die modernen Universitätseinrichtungen bildeten einen faszinierenden Kontrast, der sie in seinen Bann zog. Als sie sich durch die geschäftigen Gänge der Universität drängte, kämpfte sie mit einem Gefühl von Heimweh und Unsicherheit. Alles war so anders als in ihrer kleinen Heimatstadt. Doch dann traf sie auf Lena, ihre Mitbewohnerin, deren lebenslustige Art sofort ansteckend war. Lena schien die perfekte Begleiterin zu sein, um sich in dieser neuen Umgebung zurechtzufinden. Gemeinsam erkundeten sie die Campuscafeteria und lachten über ihre anfänglichen Ängste. Während sie sich unterhielten, bemerkte Emma einen charmanten jungen Mann, der an einem der Tische saß. Sein Name war Max, und sein selbstbewusstes Auftreten faszinierte sie. Doch sie war unsicher, wie sie mit ihren aufkeimenden Gefühlen umgehen sollte. Die Ankunft in der neuen Stadt war der Beginn eines aufregenden Abenteuers, das Emma vor Herausforderungen stellen und sie zu persönlichem Wachstum führen würde. Emma spürte eine Mischung aus Aufregung und Nervosität, als sie sich auf die bevorstehenden Prüfungen vorbereitete. Trotz ihrer Unsicherheiten in Bezug auf ihre akademischen Fähigkeiten, fand sie Unterstützung in ihrer Freundschaft mit Lena. Lena ermutigte Emma, sich neuen Herausforderungen zu stellen, und die beiden stärkten ihre Bindung, während sie gemeinsam durch lange Nächte des Lernens gingen. Inmitten des Prüfungsstress fand Emma auch Trost in den Gesprächen mit Max, der seine eigene verletzliche Seite zeigte und ihr half, sich zu entspannen. Doch Konflikte und Unsicherheiten stellten ihre Beziehung auf die Probe, als die Spannungen zwischen akademischen Anforderungen und sozialem Leben zunahmen. Trotzdem war Emma entschlossen, nach Balance in ihrem Leben zu suchen und persönliches Wachstum zu erleben, während sie sich auf die aufregende Reise des Studentenlebens einließ. Während Emma sich mit ihren aufkeimenden Gefühlen für Max auseinandersetzte, fand sie sich in einem Strudel der Unsicherheit und Aufregung wieder. Sie verbrachte Stunden damit, über ihre Begegnung mit Max nachzudenken und fragte sich, ob er ähnliche Gefühle für sie empfand. Inmitten dieser Gedanken stieß sie auf Lena, die sofort bemerkte, dass etwas Emma beschäftigte. 'Was ist los, Emma? Du siehst so nachdenklich aus', bemerkte Lena mitfühlend. Emma zögerte zunächst, ihre Gefühle preiszugeben, aber schließlich erzählte sie Lena von ihrer Begegnung mit Max und den verwirrenden Emotionen, die sie dabei empfand. Lena hörte aufmerksam zu und lächelte dann sanft. 'Manchmal ist es gut, sich dem Unbekannten hinzugeben und zu sehen, wohin es führt', ermutigte sie Emma. Diese Worte hallten in Emmas Gedanken wider, als sie sich entschloss, Max eine Nachricht zu schreiben und sich auf ein Treffen zu verabreden. Die Vorfreude und Nervosität mischten sich in ihrem Herzen, als sie auf eine Antwort von Max wartete, und sie wusste, dass dieser Schritt sie auf eine Reise des persönlichen Wachstums und der Selbstentdeckung führen würde.

\section{ Prüfungen und Erkenntnisse }
 Die Sonnenstadt Universität war in einen geschäftigen Prüfungszeitraum eingetaucht, und Emma fühlte sich von der akademischen Herausforderung überwältigt. Trotz ihres Fleißes und ihrer Hingabe fühlte sie sich unsicher in Bezug auf ihre akademischen Fähigkeiten. Als sie sich mit ihren Unterlagen in der Campusbibliothek vertiefte, spürte sie plötzlich eine vertraute Hand auf ihrer Schulter. Es war Lena, die mit einem breiten Lächeln neben ihr saß. „Hey Emma, wie läuft es?“, fragte Lena voller Mitgefühl. Emma seufzte und gestand ihre Zweifel. Lena hörte aufmerksam zu und ermutigte Emma, sich neuen Herausforderungen zu stellen. „Du bist klug und talentiert, Emma. Vertrau mehr in dich selbst“, ermutigte Lena sie. Diese Worte hallten in Emmas Gedanken wider, als sie sich auf den Weg zu ihrem nächsten Kurs machte. Die Unterstützung ihrer Freundin gab ihr neuen Mut. Während einer kurzen Pause zwischen den Vorlesungen sah sie Max auf dem Campus. Sie spürte, wie sich ihr Herz schneller schlug, als er auf sie zukam. „Hey Emma, ich wollte dich fragen, ob du Lust hast, nach den Prüfungen zusammen Zeit zu verbringen“, sagte Max mit einem warmen Lächeln. Emma lächelte verlegen und stimmte zu. Doch bald mussten sie sich den Herausforderungen stellen, die ihre Beziehung auf die Probe stellen würden. Die Prüfungen und Erkenntnisse des Studentenlebens würden Emma und ihre Freunde auf eine Reise des persönlichen Wachstums und der Selbstentdeckung führen. Die Prüfungszeit an der Sonnenstadt Universität war in vollem Gange, und Emma fühlte sich von den akademischen Anforderungen überwältigt. Trotz ihres Fleißes und ihrer Hingabe plagten sie Selbstzweifel in Bezug auf ihre Fähigkeiten. Als sie sich mit Lena in ihrem Lieblingscafé traf, um sich auf die bevorstehenden Prüfungen vorzubereiten, spiegelten sich Emmas Sorgen in ihren Augen wider. Lena bemerkte sofort die Anspannung ihrer Freundin und legte beruhigend ihre Hand auf Emmas Arm. 'Du schaffst das, Emma. Du bist klug und talentiert. Vertrau auf dich selbst', ermutigte Lena sie. Diese Worte gaben Emma neuen Mut, und sie beschloss, sich ihren Ängsten zu stellen. Während sie sich gemeinsam auf die Prüfungen vorbereiteten, erkannte Emma, wie wichtig es war, ein Gleichgewicht zwischen akademischen Anforderungen und sozialem Leben zu finden. Lena ermutigte sie, auch außerhalb des Studiums nach persönlichem Wachstum zu streben und ihre Leidenschaften zu entdecken. Inmitten des Prüfungsstresses fand Emma Trost und Unterstützung in ihrer Freundschaft mit Lena. Doch auch ihre Beziehung zu Max entwickelte sich weiter. Trotz einiger Konflikte und Unsicherheiten öffneten sich beide füreinander und lernten, authentisch zu sein. Max zeigte seine verletzliche Seite und ermutigte Emma, ebenfalls ihre Ängste und Unsicherheiten zu teilen. Die Prüfungen brachten nicht nur akademische Herausforderungen, sondern auch Erkenntnisse über sich selbst und ihre Beziehungen. Emma erkannte, dass persönliches Wachstum und Balance im Leben genauso wichtig waren wie akademischer Erfolg. Die Prüfungen und Erkenntnisse führten sie auf einen Weg des inneren Wachstums und der Selbstfindung. Max und Emma saßen in der gemütlichen Campus-Cafeteria, umgeben von lebhaften Gesprächen und dem Duft von frischem Kaffee. Emma spürte eine Mischung aus Aufregung und Unsicherheit, als sie Max gegenübersaß. Sie konnte sehen, dass er versuchte, seine eigene Verletzlichkeit hinter einem selbstbewussten Äußeren zu verbergen. 'Max, ich habe das Gefühl, dass wir uns näherkommen, aber ich spüre auch, dass du etwas vor mir verbirgst', sagte Emma mit einem Hauch von Verletzlichkeit in ihrer Stimme. Max sah sie einen Moment lang schweigend an, bevor er langsam nickte. 'Du hast recht, Emma. Ich habe mich so sehr darauf konzentriert, ein bestimmtes Bild von mir zu präsentieren, dass ich vergessen habe, wie es ist, einfach ich selbst zu sein', gestand er. Die Ehrlichkeit in Max' Stimme berührte Emma, und sie spürte, wie sich eine tiefere Verbindung zwischen ihnen zu entwickeln begann. 'Es ist okay, Max. Wir alle haben unsere Unsicherheiten und Ängste. Aber ich möchte dich kennenlernen, den echten Max', ermutigte Emma ihn. In diesem Moment fühlten sie sich beide freier und offener, bereit, sich auf die authentische und tiefe Verbindung einzulassen, die zwischen ihnen entstehen konnte.

\clearpage

% Metadata
\section*{Metadaten}
\colorbox{lightgray}{
    \begin{minipage}{\dimexpr\textwidth-2\fboxsep}
        \vspace{1cm}
        \begin{itemize}
            \item Name des Buches: Studentleben
            \item Name des Autors: Maja Schmidt
            \item Name des Herausgebers: Mark Zimmermann
            \item Name des Verlags: HdM AI Technologies
            \item Adresse des Verlags: Nobelstraße 10, 70569 Stuttgart
            \item Datum der Veröffentlichung: 2022-10-20
        \end{itemize}
        \vspace{1cm}
    \end{minipage}
}

\end{document}